\section{Birth and Evolution of an Optical Vortex \cite{ValBE16}}
\label{sec:birth_evol}
The paper is proposed to describe the evolution of an optical vortex beam passing through a generic optical system composed of free space and two lenses. Recently, a very general soltuion to the paraxial wave equation is introduced \cite{Ban08CiB}. This family of solutions is called circular beams (CiB) and contains many well known beams carrying orbital angular momentum (OAM) as special cases with specific beam parameters. The main advantage using CiB to describe a vortex beam is the simple transformation law under a generic ABCD system, enabling us to theoretically and physically model the propagation of vortices through paraxial optical system.
\subsection{Circular beam}
To obtain the general CiB solution to the paraxial wave equation (Appendix III)
\begin{eqnarray}
	\left(\nabla_{\perp}^2 + 2ik \frac{\partial}{\partial z} \right) U(\bar{r}, z) = 0,
\end{eqnarray}
where $\nabla_{\perp}^2 = \partial^2 / \partial x^2 + \partial^2 / \partial y^2$ is the transverse Laplacian and $\bar{r} = \hat{x} x + \hat{y} y$, we shall perform the seperation of variables
\begin{eqnarray}
	U(\bar{r}, z) = Z(z) F(u, v) {\rm GB}(r, q),
\end{eqnarray}
where $(u, v) = (x, y)/\chi(z)$ is a scaled Cartesian coordinate by an unknown $z$-dependent factor $\chi(z)$ and ${\rm GB}(r, q)$ represents the fundamental Gaussian beam: ${\rm GB}(r, q) = (1 + z/q_0)^{-1} \exp[ i k r^2/ 2q(z)]$ with $q_0$ being the complex beam parameter at $z = 0$ and $q(z) = z + q_0$.